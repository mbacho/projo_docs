\chapter{Introduction}
\section{Background}

\noindent
The World Wide Web is an expansive system of interlinked information including text, audio and visual media. This information is stored in web servers and accessed using tools such as web browsers, provided there is a working connection from the web browser to the web server e.g. an internet connection. A website is a collection of related web pages served from a single web domain hosted by one or more web servers. The World Wide Web is composed of all websites that are available publicly. \cite[p. 3]{aguillo2008} 

\noindent
In today's era of the internet, websites have become a major avenue through which information is extracted, published and distributed. Websites range from simple personal pages to complex institution pages to fully fledged web applications. Companies, institutions and individuals use to websites to post information about themselves, topics of interest, ideas e.t.c. Websites have become so integrated within the society that they are now being used to achieve more than just information sharing. Reasons for developing websites are as many as the uses of websites e.g. communication, information sharing, business, banking, entertainment, gaming and many more.

\noindent
With every passing day, more and more researchers, students and interested parties turn to the world wide web to satisfy their information thirst. Whether it's to find some information about a simple issue or find research information, many more people are relying on internet sources. However, the credibility of internet sources is left to the discretion of the user. Thus, a user may find wrong information from the web and use it. Therefore, there is a need to highlight and improve the credible sources of information, especially academic and scholarly information. One major source of academic and scholarly information is academic institutions. They provide accurate and credible scholarly information. Most institutions of higher learning have an online presence, mostly in the form of institution websites. Such websites provide information to the general public on a wide array of topics.

\noindent
The release of such information from academic institutions helps to impact the target audience in a positive manner. Some of the information published in academic websites is targeted at a certain audience. For example, a paper about nuclear physics published in an academic website would be targeted at individuals interested in nuclear physics. However, the information providers may not know if the information is effective in achieving their goal of information dispersing to their target audience. Thus, website owners and administrators need a tool to help them assess the impact of their website. This project aims to help website owners to analyze and determine the effectiveness of their websites.


\section{Problem Statement}
This project will focus on websites from academic institutions in Kenya. An academic institution's website is particularly important since it should reflect the academic excellence of the institution. Such institutions may have set out guidelines and policies governing their websites but don't have effective and efficient tools to implement the policies or measure website improvement over time. 

\noindent
Take the University of Nairobi for example. It has a system of websites under the domain 'uonbi.ac.ke'. The university has a system of websites under the domain 'uonbi.ac.ke'. Under the domain 'uonbi.ac.ke', units within the  University of Nairobi have sub domains. These units include colleges, schools, faculties, institutes, centers, university staff and student organizations and university service providers. The system of websites is governed by a policy. According to the university's website policy, the main objectives for the website are:
\begin{itemize}
\item To ensure accuracy, consistency, integrity of the content and protection of the identity and image of the University.
\item To improve the University's visibility regionally and internationally and create a strong brand in line with the University's Strategic Plan.
\item To provide a set of mandatory guidelines for the University of Nairobi System of Websites
\item To guide the maintenance of the web content and evolution of the System of Websites to ensure continued reflection of the true status of the University within its web space.
\end{itemize}

\noindent
There has been a need for analysis of academic institutions' websites so as to know the effectiveness of the websites, their impact and how to improve them. Currently, there is no effective method to analyze and determine the effectiveness and impact of the system of websites by the university itself. This is especially important to learning institutions since they need to know the impact they have on their intended audience.  Such analysis would be helpful in determining whether the system of websites is upholding the set policy.


\section{Project Justification}
\noindent
Provision of information about the effectiveness and impact of websites will enable website administrators to make more informed decisions. For example, they would know when to improve on their website or policies governing the same.
\begin{review_comment}{Do it}{red}
{Add webometrics objectives}
\end{review_comment}


\section{Assumptions}
Assumptions made during the development of the project include:
\begin{itemize}
\item The universities used in the analysis are Kenyan and have a web presence
\end{itemize}
